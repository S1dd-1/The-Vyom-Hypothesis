\documentclass{article}
\usepackage{graphicx}
\usepackage{parskip}
\usepackage{arevmath}
\usepackage{amsmath}
\usepackage{sectsty}
\usepackage{libertinust1math}
\usepackage{multirow}
\usepackage{array}

\usepackage{biblatex}
\usepackage{hyperref}
\addbibresource{main.bib}

\title{The Vyom Hypothesis}
\author{Siddarth S. Nair}
\date{May 2025}



\begin{document}
\begin{titlepage}
    \centering
    \vspace*{\fill}
    {\Huge\bfseries The Vyom Hypothesis \par}
    \vspace{1cm}
    {\Large Siddarth S. Nair \par}
    \vspace{0.5cm}
    {\large May, 2025 \par}
    \vspace*{\fill}
    \thispagestyle{empty}
\end{titlepage}



\newpage
\sectionfont{\fontsize{23}{22}\bfseries}
\subsectionfont{\fontsize{15}{22}\bfseries}



\begin{abstract}

The universe emerged from nothing, and thus must ultimately return to it.
This paper introduces the concept of existence ($\Koppa$), defined as the measure of an entity’s complexity and interactability, and proposes that all entities tend towards a state of non-existence.
Following this Occam's Razer inspired postulate, this speculative framework attempts to unify diverse physical phenomena through intrinsic and extrinsic tendencies (pathways through which entities move toward non-existence).
Several quantifiable relations are reinterpreted from this perspective:
(a) Newton’s inverse-square law is derived from extrinsic tendencies, suggesting that fundamental forces are emergent;
(b) the accelerated expansion of space-time appears natural as described by its own intrinsic tendency; and
(c) the inverse-area dependence of Hawking radiation power can be interpreted as an extrinsic tendency acting upon a black hole by surrounding space-time, and increasing as the surface area decreases. 
However, this work remains an exploratory foundation; many characteristics of entities within this framework are still speculative or incomplete.

\end{abstract}



\newpage
\tableofcontents



\newpage
\section{Shortcomings}


As the author is currently a high school student, this paper reflects a developing understanding of advanced physics. 
While the Vyom Hypothesis proposes a unifying framework, many aspects remain unexplained, speculative, or incomplete. 
Further education and years of work will be necessary to rigorously formulate, test, and potentially validate this hypothesis. 
This paper and all that it entails is intended as a foundational exploration rather than a finalized model. 
The author appreciates the reader’s patience and understanding.



\newpage
\section{Postulate}


\vspace{0.4cm}
\begin{center}
All entities tend toward non-existence.
\end{center}


\subsection{Existence}

Existence ($\Koppa$) is the quantitative measure of a system’s capacity to participate in and generate interactions with other entities, arising from the combined effects of its complexity and its interactability. 
In other words, its entropy ($S$) and interaction potential ($U$) respectively. 
Mathematically it can be represented as
\[
    \Koppa \propto US\;.
\]

Non-existence refers to a state where the entity does not interact with any other entities, not even itself. Minimally existent states also exist, as observed in a lone particle interacting solely with spacetime in the heat death scenario.


\subsection{Entity}

Entities are the objects of interest; they may refer to singular particles, celestial bodies, or any defined region of study. 
A system is a collection of entities that interact with one another. 
For the purpose of analysis, a system may be treated as isolated from external influences and can itself be regarded as an entity.


\subsection{Interaction Potential}

Tendencies are the pathways that entities follow to reach non-existence. 
Interaction potential measures the capacity of an entity to approach non-existence through these tendencies. 
Each tendency can be either intrinsic, where the entity approaches through its own behavior, or extrinsic, where it does so through external factors such as the surrounding medium or other entities.
The total interaction potential is given by
\[
    U = \sum_i U_i\;,
\]
where $U$ is the net interaction potential and $U_i$ represents an individual intrinsic or extrinsic interaction potential. 

Tendency is the rate of change of the interaction potential with respect to distance and is expressed as
\[
    T = \frac{dU}{dr}\;.
\]


\subsection{Entropy}

Entropy is a measure of disorder or randomness within a system. 
All systems naturally evolve toward a state of maximum entropy, where energy is distributed as evenly as possible. 

In this framework, increasing entropy initially corresponds to greater apparent existence, since increased randomness allows for a wider range of interaction potentials between sub-entities. 
However, as the system approaches maximum entropy, these interactions diminish in strength; energy becomes too evenly distributed, tendencies weaken, and existence falls toward a minimum. 
At the largest scale, this state aligns with non-existence, as subsystems become effectively isolated from one another.

For the purposes of analysis, entropy can be expressed through the number of microstates of an entity that is consistent with a macrostate, as described by Ludwig Boltzmann \cite{BoltzmannTranslation2015}:
\[
    S = k_b \ln \Omega\;.
\]



\newpage
\section{Case Studies}

All entities within a system either attract or repel other entities through extrinsic tendencies, whether through gravity, electromagnetism, or the strong nuclear force. 
Even intrinsic tendencies ultimately manifest through similar characteristics, as seen in phenomena such as the weak nuclear force, or self-induced gravity. 
Different entities are examined in the following subsections.


\subsection{Matter}

Gravity emerges as matter's most efficient tendency towards non-existence. 
On intermediate scales, it increases existence by driving subsystems into greater interactability. 
However, once the concentration of matter exceeds the Schwarzschild condition, the system loses all distinguishable subsystems and collapses into a black hole, as will be explored later.


\subsection{Spacetime}

Spacetime’s tendency arises from the presence of regular matter. 
Since spacetime is the medium through which gravitational interactions occur, it too seeks non-existence by reducing these interactions. 
To achieve this, spacetime expands, causing the interactions between distant entities 
to become negligible and effectively isolating them from one another.
This also acts as an extrinsic tendency for photons, causing them to redshift into oblivion.


\subsection{Electromagnetism}

Electromagnetism is a special case among the fundamental forces, as the same tendency can act as either an attractor or a repellent.

Electrostatic tendencies arise from charge imbalances between entities. 
Like charges naturally repel, since moving apart reduces the overall charge complexity. 
Conversely, unlike charges attract as they move toward annihilation and non-existence. 
Protons and electrons follow this pathway, though Heisenberg’s uncertainty principle prevents the electron from collapsing into the nucleus within atoms \cite{Heisenberg2007}.

When charges begin to move, however, new tendencies emerge. 
A moving charge generates a magnetic field, which influences other moving charges through the Lorentz force. 
Unlike the direct attraction or repulsion seen in electrostatics, magnetic effects act perpendicularly to both the motion of the charge and the field itself. 
This causes trajectories to bend, redistributing energy without direct annihilation. 

On larger scales, the interdependence of electric and magnetic fields produces self-propagating electromagnetic waves, photons that carry energy outward through spacetime, until they eventually they reach non-existence as aforementioned.


\subsection{Strong and Weak Nuclear Forces}

The strong nuclear force acts as a tendency that clumps entities together, binding quarks into nucleons and nucleons into stable nuclei. 
By eliminating free sub-entities, it reduces interactability and moves the system closer to non-existence. 

The weak nuclear force, on the other hand, manifests as a tendency toward dissolution, or radioactive decay, transforming unstable entities into more fundamental components.  

Both nuclear forces can therefore be understood as intrinsic tendencies that drive matter either toward condensation or decomposition, consistent with the broader framework.


\subsection{Black Holes}

Within this framework, black holes can be regarded as regions of non-existence. 
Matter follows gravitational tendencies to condense into an extremely dense state, such that all entities in it become indistinguishable, representing a minimal state of existence. 

The event horizon, however, remains part of the existent universe. 
It acts as a boundary separating existence from non-existence and carries entropy encoded upon its surface, as described by the Bekenstein–Hawking holographic principle \cite{ferrari2025bekensteinhawkingentropybriefoverview}.

Hawking radiation represents the extrinsic tendency acting upon the event horizon by the spacetime around it \cite{HawkingRadiation2005}.
It arises from quantum fluctuations, where a particle-antiparticle pair fails to annihilate and become real particles. One falls into the black hole and the other moves away, thus radiating away energy.
As the black hole evaporates, the interior region of non-existence contracts, while the emitted radiation disperses through spacetime, contributing to the overall isolation of energy and matter. 
Thus, while the local domain of non-existence decreases, the outward non-existence, the loss of interaction between entities separated by vast distances, continues to grow.



\newpage
\section{Mathematical Formulation}


\subsection{Existence}

Consider a system of many entities acting under their tendencies. 
Its existence can be expressed as
\begin{equation*}
    \Koppa \propto US\;,
    \qquad
    \Koppa = \alpha US\;.
\end{equation*}
Where $\Koppa$ denotes existence, $\alpha$ is an existence factor, $U$ is the net interaction potential, and $S$ represents entropy of the system.

Interaction potential is the sum of extrinsic and instrinsic potentials.
Extrinsic potentials are inversely proportional to distance between the entity and the source, while intrinsic potentials vary from case to case.
\begin{equation*}
    U = \sum \frac{\beta_{i}}{r_{i}} + \sum u_{i}\;.
\end{equation*}
Where $U$ denotes interaction potential acting on an entity, $\beta_{i}$ and $r_{i}$ are the specific factor and distance of an extrinsic interaction source respectively, and $u_{i}$ represents intrinsic potentials.

For the purposes of this derivation, we can represent entropy, $S$, in terms of the number of microstates in a macrostate, $\Omega$. $k_{b}$ denotes the Boltzmann constant. 
In other words,
\begin{equation*}
    S = k_{b} \ln \Omega\;.
\end{equation*}

Thus, existence can be expanded as:
\begin{equation*}
    \Koppa = \alpha \left( \sum \frac{\beta_{i}}{r_{i}} + \sum u_{i} \right) \left( k_{b} \ln \Omega \right)\;.
\end{equation*}
If we consider entities acting solely under extrinsic potentials, then 
\begin{equation*}
    \Koppa = \alpha \left( \sum \frac{\beta_{i}}{r_{i}} \right) \left( k_{b} \ln \Omega \right)\;.
\end{equation*}
Which on differentiating with respect to time gives:
\begin{equation*}
    \frac{d\Koppa}{dt} = \alpha \left( -\sum v_{i}\frac{\beta_{i}}{r^2_{i}} \right) \left( k_{b} \ln \Omega \right)\;,
\end{equation*}
where, $v_i = \frac{dr_i}{dt}$ is the radial velocity; it is negative for attraction and positive for repulsion.

From this, several inferences can be made:

\begin{itemize}
    \item The rate of change of existence mirrors Newton's inverse-square law for extrinsic tendencies.
    \item Whether existence decreases or increases depends upon whether entities are attracting or repelling away ($v_{i}$) and the specific factor ($\beta_{i}$).
    \item Minimal existence is reached when objects attract each other and collapse into extremely dense entities ($ \Omega \rightarrow 1,\; \ln \Omega \rightarrow 0$) or when they disperse infinitely apart ($r \rightarrow \infty$).
    \item Fundamental forces are emergent behaviors of tendencies.
\end{itemize}


\subsection{Emergent Behaviour of Extrinsic Tendencies}

As discussed, fundamental forces can be understood as emergent manifestations of tendencies.
Their observed laws arise naturally from the form of $U$ and its time evolution,
\[
    \frac{dU}{dt} = -\sum v_{i}\frac{\beta_{i}}{r_{i}^2}\;,
    \qquad
    \frac{dU}{dr} = -\sum \frac{\beta_{i}}{r_{i}^2}\;.
\]
This is directly analogous to the classical definition of net force as the spatial derivative of potential energy:
\[
    F = -\frac{dU}{dr}\;.
\]
The present framework establishes this analogy but does not yet provide a complete derivation of all forces; it remains as the prime candidate for future development.


\subsection{Black Holes}

Hawking radiation arises from the extrinsic tendency of spacetime. 
Accordingly, the tendency is inversely proportional to the square of the distance between the singularity and surrounding spacetime; namely, the radius of the black hole.
\[
    \frac{dU}{dt} \propto -\frac{1}{r^2}\;,
\]
where $r$ is the radius. Equivalently,
\[
    \frac{dU}{dt} \propto -\frac{1}{A}\;.
\]
Where $A$ is the surface area of the black hole. 
This directly mirrors Stephen Hawking's conclusion that the intensity of Hawking radiation is inversely proportional to the surface area of the event horizon. This, however, does not constitute a rigorous proof.



\newpage
\section{Further Speculation}


\subsection{Decay of Fermions}

In the Standard Model, fermions such as protons, electrons, and neutrinos are highly stable and are not known to decay. 
However, several grand unified theories propose that these particles may have extremely long lifetimes: on the order of $10^{34}$ years for protons in particular.
 
Spacetime already expands to reduce gravitational interactions between entities, thus it must also interact with the entities directly.
It may provide a pathway for fundamental particles to approach non-existence through its own intrinsic tendency.


\subsection{Dark Matter's limited interactability}

Dark matter could be regions of minimal-existence that do not interact electromagnetically unlike black holes, yet still result in the stability of galaxies through gravity.
However, individual entities would need to be extremely small, which may explain why none have been detected experimentally.


\subsection{The Universe's Origins}

The universe presumably emerged from nothing; in other words, the Big Bang was a spontaneous process that increased existence by an enormous factor. 
One possibility is that, after all matter dissolves into energy through Hawking radiation or the aforementioned fermion decay, the expansion of spacetime could eventually reverse. 
This would arise from an attractive tendency to reduce quantum complexity. 
A state similar to that inside a black hole would then be reached, a configuration of indistinguishable entities. 
However, since spacetime itself is the only remaining entity, the conditions required for one are not fulfilled. 
A low-entropy state is reached once more, resetting the stage for the next Big Bang.



\newpage
\printbibliography



\end{document}
